%----------------------------------------------------------------------------------------
%	SECTION TITLE
%----------------------------------------------------------------------------------------

\cvsection{Academic Projects}

%----------------------------------------------------------------------------------------
%	SECTION CONTENT
%----------------------------------------------------------------------------------------

\begin{cventries}

%------------------------------------------------

\cventry
{C, Vectorization, CUDA} % Technologies
  {Dotprod} % Title
{\href{https://github.com/JoseFilipeFerreira/dotprod}{GitHub Repository}} % Link
{} % Grade
{ % Description(s)
\begin{cvitems}
\item {Matrix Multiplication Optimization exploring Memory Acess tuning and Vectorization}
\item {Parallelization of the algorithm for heterogeneous servers with NVidia GPUs, using CUDA}
\end{cvitems}
}

%------------------------------------------------

\cventry
{C++, Paralelization} % Technologies
  {Parallel Raytracer} % Title
{\href{https://github.com/JoseFilipeFerreira/parallel-raytracer}{GitHub Repository}} % Link
{} % Grade
{ % Description(s)
\begin{cvitems}
\item {Raytracer Performance Optimization for heterogeneous servers, where a Bounding Volume Hierarchy data structure was tested to improve the rendering performance of a pool of parallel workers}
\end{cvitems}
}

%------------------------------------------------

\cventry
{GLSL} % Technologies
  {Terrain Generation} % Title
{\href{https://github.com/JoseFilipeFerreira/TerrainGeneration}{GitHub Repository}} % Link
{} % Grade
{ % Description()
\begin{cvitems}
\item {Terrain Generation based on OpenGL Shading Language (GLSL) and height maps}
\end{cvitems}
}

%------------------------------------------------

\cventry
{C++, OpenGL, XML} % Technologies
  {Engine} % Title
{\href{https://github.com/JoseFilipeFerreira/engine}{GitHub Repository}} % Link
{} % Grade
{ % Description(s)
\begin{cvitems}
\item {Generic Graphic Engine capable of efficiently rendering any kind of scene
  defined in a XML configuration file}
\end{cvitems}
}

%------------------------------------------------

\end{cventries}
